\documentclass{article}

% set font encoding for PDFLaTeX or XeLaTeX
\usepackage{graphicx}
\usepackage{ifxetex}
\ifxetex
  \usepackage{fontspec}
\else
  \usepackage[T1]{fontenc}
  \usepackage[utf8]{inputenc}
  \usepackage{lmodern}
\fi

% used in maketitle
\title{Actividad 3}
\author{Corral Valdez Jesus Giovanni\\
Departamento de Física
}

% Enable SageTeX to run SageMath code right inside this LaTeX file.
% documentation: http://mirrors.ctan.org/macros/latex/contrib/sagetex/sagetexpackage.pdf
% \usepackage{sagetex}

\begin{document}
\maketitle
\clearpage
\section{Posición de un proyectil}
Se puede saber la posición de un proyectil en cualquier instante de tiempo \\
conociendo la velocidad con la que fue lanzada y el ángulo de inclinación.
\subsection{Posicion horizontal}
\begin{equation}
x = v_o t cos(\theta)
\end{equation}
Con esta ecuacion se puede encontrar donde está en el eje horizontal la partícula, o mas bien, se puede saber el desplazamiento que ha llevado.
\subsection{Posición vertical}
\begin{equation}
y = v_o t sin(\theta) - \frac {1} {2} gt^2
\end{equation}
Esta ecuación te proporciona la ubicación de la partícula en el eje vertical
\section{Programa para encontrar la posición}
\begin{verbatim}
program ms
  implicit none
  integer :: i, a
  integer, parameter :: ntimes = 100
  integer, parameter :: maxang = 90
  real, dimension (200) :: x,y
  real :: radian, time, fa, fi
  real, parameter :: deltat = 0.1
  real, parameter :: g = 5  !como ocupamos 1/2 g para los calculos.
  real, parameter :: pi = 3.1415927
  real, parameter :: vo = 10


 open (1, file = 'losdatos.dat', status='unknown')
  do a=15, 90, 15
	fa = float(a)
	radian = fa * pi / 180
	do i=1, ntimes
	fi = float(i)
	time = fi * deltat
	x(i) = vo * time * cos(radian)
	y(i) = vo * time * sin(radian) - 0.5 * g * time * time
	if(y(i).LT.0) exit	
   	write (1,*) x(i), y(i)


  end do
write(1,*)' '
end do
close(1)


end program ms
   


!!Este programa calcula la posicion de una particula la ser lanzada
!!con una velocidad de 10 m/s y a distintos angulos predeterminados.

\end{verbatim}
Este programa proporciona a una hoja de datos los valores de la posición tanto horizontal  como vertical del proectil cuando es lanzado a 10 m/s a distintos angulos (15, 30, 45, 60, 75, 90).
\clearpage
\section{Gráfica}
Esta es la gáfica que da expone los datos obtenidos durante la ejecución del programa realizado en esta práctica.
Se realizado con el graficador GnuPlot con el código:
\begin{verbatim}
plot 'losdatos.dat' with linespoints ls 1

\end{verbatim}
\begin{figure}
  \includegraphics[width=\linewidth]{datos.png}
  \caption{Grafica de posición.}
  \label{fig:boat1}
\end{figure}

\end{document}

\end{document}
\documentclass{article}

% set font encoding for PDFLaTeX or XeLaTeX
\usepackage{hyperref}
\usepackage{ifxetex}
\ifxetex
  \usepackage{fontspec}
\else
  \usepackage[T1]{fontenc}
  \usepackage[utf8]{inputenc}
  \usepackage{lmodern}
\fi

% used in maketitle
\title{Actividad 2}
\author{Corral Valdez Jesus Giovanni\\
Departamento de Física\\
Universidad de Sonora}

% Enable SageTeX to run SageMath code right inside this LaTeX file.
% documentation: http://mirrors.ctan.org/macros/latex/contrib/sagetex/sagetexpackage.pdf
% \usepackage{sagetex}

\begin{document}
\maketitle
\clearpage
\section{Movimiento de proyectiles.}
El movimiento de proyectiles es un tipo de movimiento en el que una partícula u objeto es lanzada cerca de la superficie de la Tierra y se mueve a través de un recorrido curveado bajo la acción de la gravedad. El objeto comienza con una velocidad inicial que depende de las componentes de la velocidad horizontal y vertical que tenga, que van cambiando conforme va pasando el tiempo en función de la aceleración y la fuerza de gravedad.

\clearpage
\section{Programas para calculo de proyectiles}
\subsection{Primer Programa: Calculo de Proyectiles.}
En la tabla de de la siguiente hoja se muestra una serie de datos que se le proporcionó al programa y los datos que este arrojó, este programa de ejemplo calcula la posición vertical y horizontal, así como la velocidad final. Pero no es posible encontrar con certeza el tiempo necesario para encontrar el alcance maximo con el angulo de 45 grados.
\begin{verbatim}
program projectile
  implicit none

  ! definimos constantes
  real, parameter :: g = 9.8
  real, parameter :: pi = 3.1415927

  ! definimos las variables
  real :: a, t, u, x, y
  real :: theta, v, vx, vy

  ! Leer valores para el ángulo a, el tiempo t, y la velocidad inicial u desde la terminal
  write(*,*) 'Dame el ángulo, el tiempo y la rapidez inicial'
  read(*,*) a, t, u

  ! convirtiendo ángulo a radianes
  a = a * pi / 180.0
  
  ! las ecuaciones de la posición en x y y
  x = u * cos(a) * t
  y = u * sin(a) * t - 0.5 * g * t * t

  ! La velocidad al tiempo t
  vx = u * cos(a)
  vy = u * sin(a) - g * t
  v = sqrt(vx * vx + vy * vy)
  theta = atan(vy / vx) * 180.0 / pi
 
 ! escribiendo el resultado en la pantalla
  write(*,*) 'x: ',x,'  y: ',y
  write(*,*) 'v: ',v,'  theta: ',theta

end program projectile


\end{verbatim}
\begin{table}[]
\centering
\caption{Calculo de proyectiles.}
\label{my-label}
\begin{tabular}{llllll}
ang  & t  & Vo & x        & y          & Vf       \\
50 & 40 & 30 & 771.3452 & -6920.7466 & 369.5222 \\
80 & 4  & 45 & 31.2567  & 98.8654    & 9.3401   \\
45 & 10 & 30 & 245.7456 & -317.9271  & 84.4474 
\end{tabular}
\end{table}


\subsection{Segundo Programa: Tiempo de Vuelo.}
\begin {equation}
t = \frac {2v_o sin(theta)} {g}
\end {equation}
\begin{verbatim}
program tiempo_vuelo
  implicit none
  !definimos constantes que se utilizaran
  real, parameter :: g = 9.8
  real, parameter :: pi=3.1415927

  !se define las variables
  real :: a, v
  real :: t

  !Leer valores para el angulo y velocidad inicial.
  write(*,*) 'Diga angulo, y la velocdad inicial'
  read(*,*) a, v

  !Convertir los grados a radianes.
  a= a * pi / 180.0

  !la ecuacion para encontrar el tiempo.
  t= (2 * v) * sin(a)
  t= t / g

  !El resultado del tiempo.
  write(*,*) 't: ',t

 end program tiempo_vuelo
\end{verbatim}

\clearpage
\subsection{Tercer programa: Altura Maxima}
\begin {equation}
h = \frac {v^2_o sin^2(theta)} {2g}
\end {equation}
\begin{verbatim}
program altura_max
  implicit none
  real, parameter :: g = 9.8
  real, parameter :: pi = 3.1415927
  real :: v, a
  real :: h, square, two !altura, cuadrado del seno, dos veces la
                         !gravedad.
  

  !leer valores para la velocidad inicial y el angulo de la
  !trayectoria.
  write (*,*) 'De la velocidad inicial y el angulo de tiro'
  read (*,*) v, a

  !convertir el angulo a radianes.
  a = a * pi / 180


  !sacar la altura maxima.
  v = v * v !velocidad al cuadrado
  square = sin(a) * sin(a) !seno al cuadrado
  two = 2 * g
  h = v * square
  h = h / two

  !el resultado
  write(*,*) 'altura maxima: ',h


end program altura_max
\end{verbatim}

\clearpage

\subsection{Cuarto programa: Distancia Maxima}
\begin {equation}
d = \frac {v^2_o} {g} sin(2theta)
\end {equation}
\begin{verbatim}
program x_max
  implicit none

  real, parameter :: g = 9.8
  real, parameter :: pi = 3.1415972

  real :: v, a
  real :: square, two, d !cuadrado de la velocidad, dos veces el
                         !angulo, distancia final. 

  write(*,*) 'Indique la velocidad inicial y el angulo de tiro: '
  read(*,*) v, a


  !sacar radianes
  a = a * pi /180
  
  !sacar la velocidad al cuadrado y esta entre gravedad.
  v = v * v
  v = v / g

  !el doble angulo.
  two = 2 * a

  !el calculo de la distancia.
  d = v * sin(two)

  write (*,*) 'La distancia maxima de tiro es: ', d

end program x_max
  
  
\end{verbatim}

\clearpage

\section{Bibliografia}
\url{https://en.wikipedia.org/wiki/Projectile_motion}


\end{document}

\documentclass{article}

% set font encoding for PDFLaTeX or XeLaTeX
\usepackage{ifxetex}
\ifxetex
  \usepackage{fontspec}
\else
  \usepackage[T1]{fontenc}
  \usepackage[utf8]{inputenc}
  \usepackage{lmodern}
\fi

% used in maketitle
\title{Evaluacion 1}
\author{Corral Valdez Jesus Giovanni\\
Departamento de Fisica \\
Universidad de Sonora}
\date{30 de octubre de 2017}


% Enable SageTeX to run SageMath code right inside this LaTeX file.
% documentation: http://mirrors.ctan.org/macros/latex/contrib/sagetex/sagetexpackage.pdf
% \usepackage{sagetex}

\begin{document}
\maketitle
\clearpage

\section{Actividad 1: Esfera}
\begin{verbatim}
program esfera


  implicit none  

  integer :: ierr
  character(1) :: yn
  real :: radius, area, volumen
  real, parameter :: pi = 3.141592653589793

  interactive_loop: do


    write (*,*) 'Declare el radio de la esfera'
    read (*,*,iostat=ierr) radius

    if (ierr /= 0) then
      write(*,*) 'Error, entrada invalida'
      cycle interactive_loop
    end if


    area = 4 * pi * radius * radius
    volumen = 4 * pi * radius**3 / 3


    write (*,'(1x,a7,f14.2,5x,a7,f14.2,5x,a9,f14.2)') &
      'radius=',radius,'area=',area, 'volumen=',volumen

    yn = ' '
    yn_loop: do
      write(*,*) 'Perform another calculation? s[n]'
      read(*,'(a1)') yn
      if (yn=='s' .or. yn=='S') exit yn_loop
      if (yn=='n' .or. yn=='N' .or. yn==' ') exit interactive_loop
    end do yn_loop

  end do interactive_loop

end program esfera

\end{verbatim}

\section{Actividad 2: Medias}
\begin{verbatim}
program summation
implicit none
integer :: suma, a, conta
real :: aritme, armoni, sumarmo, fa, fc, fs

print*, "Este programa realiza las medias de una sumatoria, cuando quiera aplaste 0 para terminar"
open(unit=10, file="SumData.DAT", status='unknown')

suma = 0
conta = 0
sumarmo = 0

do
 print*, "De numero:"
 read*, a
 if (a == 0) then
  exit
 else
suma = suma + a
conta = conta + 1
fa = float(a)
fa = 1/fa
sumarmo = sumarmo + fa

 end if
 write(10,*) a
end do
fs = float(suma)
fc = float(conta)
aritme = fs / fc
armoni = fc / sumarmo


print*, "Sumatoria =", suma
write(10,*) "Sumatoria =", suma
write(10,*)' '
print*, "Media aritmetica =", aritme
write(10,*) "Media aritmetica =", aritme
write(10,*) ' '
print*, "Media armonica =", armoni
write(10,*) "Media armonica =", armoni
write(10,*) ' '


close(10)

end
\end{verbatim}

\section{Actividad 3: Pi}
\begin{verbatim}
program serie
 implicit none
integer :: i
real :: n, suma, iteracion, pi

  pi = 1
  iteracion = 1
 write(*,*) 'El valor de pi/4 segun las repeticiones:'
     do i=1, 50
     iteracion = iteracion * (-1)
     n = 2 * i + 1
     n = 1 / n
     n = n * (iteracion)
     pi = pi + n
	if (i.EQ.10) then
	write(*,*) ' '
	write(*,*) '10:', pi
        end if

	if (i.EQ.20) then
	write(*,*) ' '
	write(*,*) '20:', pi
        end if

	if (i.EQ.30) then
	write(*,*) ' '
	write(*,*) '30:', pi
        end if

	if (i.EQ.40) then
	write(*,*) ' '
	write(*,*) '40:', pi
        end if

	if (i.EQ.50) then
	write(*,*) ' '
	write(*,*) '50:', pi
        end if

end do

end program serie

\end{verbatim}


\end{document}
